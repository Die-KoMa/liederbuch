\lied{Rechnen, rechnen, rechnen}{Saufen, saufen, saufen}{2007}

\strophe{ 
Es gibt Mengen, die sind mehr als mächtig. \\
Es gibt Mengen ohne was drin. \\
Es gibt Formeln ohne Symbole. \\
Es gibt Beweise ohne Sinn. \par
Es gibt Zahlen, die kann man nicht zählen. \\
Es gibt Leute, die teilen durch Null. \\
Es gibt schnelle Sekretärinnen \\
Und es gibt Tee, so gut wie Red Bull. \par
Es gibt schon so viel und es wird immer mehr, \\
Und wir können alles glauben. \\
Aber am schlimmsten ist immer noch: \\
Lösungen zu rauben.}

\refrain{
(Darum:) Rechnen, rechnen, rechnen, rechnen, \\
Rechnen, probieren, zerreißen. \\
Rechnen, rechnen, rechnen und am Ende \\
Doch noch beweisen.}
 
\strophe{
Es gibt Primzahlen, die sind grade. \\
Es gibt Sitze, die sind spitze. \\
Es gibt unentscheidbare Aussagen \\
Und unendlich schlechte Witze. \par
Es gibt Vorlesungen für Kinder. \\
Es gibt junge Professoren. \\
Es gibt Erstsemestereinführungen \\
Mit doll motivierten Tutoren. \par
Sie sagen, für Gebühren ist es nie zu spät. \\
Doch dabei verkennen sie die Realität: \\
Studenten auf der Straße fahnenschwenkend laufen. \\
Aber am besten ist immer noch: \\
Lösungen verkaufen.}

\strophe{(Refrain)}

\strophe{\textsl{Alternativer zweiter Refrain:} \\
Rechnen, rechnen, rechnen, rechnen, \\
Rechnen, streiken, agieren. \\
Rechnen, rechnen, rechnen und daneben \\
Noch protestieren.}



