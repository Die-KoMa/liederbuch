\lied{Zahlen}{Männer}{2008}

\strophe{
Zahlen machen uns arm, Zahl'n beschäftigen die Welt. \\
Zahlen können komplex sein, Zahlen können verwirrend sein. \\
Oh Zahlen sind auch gefährlich. \\
Doch Zahlen meinen es meistens ehrlich.}

\strophe{
Mit Zahlen kauft man ein, nur mit Zahlen geht alles gut. \\
Zahlen sind ziemlich öde, nur wenige ham Mut \\
Uns're Zahlen zu erfassen. \\
Auf diese Menschen hoch die Tassen!}

\refrain{
Manche Zahlen sind prim, manche ganz. \\
Zahl'n verleihen uns uns'ren Glanz. \\
Doch eins versteh' ich bis heute nicht: \\
(2x) Warum ist Pi reell? \\
Warum ist Pi nicht drei?}

\strophe{
Wenn Pi nur in $\mathbb{Z}$ wär', wäre die Welt ganz leicht. \\
Kreise hätten jetzt Ecken, hätte das nicht ausgereicht? \\
Warum muss man denn so genau sein? \\
Ein Kassenbon reicht heut' doch auch als Fahrschein!}

\strophe{(Refrain)}

\strophe{
Ganze Zahlen sind doch einfacher zu bestimm' \\
Und positiv helfen sie zu gewinn' \\
Gegen Bruchzahlen, gegen Kommata -- macht das einen Sinn?}

\refrain{
(2x) Warum ist Pi reell? \\
Warum ist Pi nicht drei?}

\strophe{
Was wäre Mathe eigentlich ohne Pi? \\
Jeder könnte rechnen, Physiker freuen sich wie nie! \\
Das könnt' man doch nicht machen -- \\
Physiker haben nichts zu lachen!}

\refrain{
Es gibt nen Grund für $\mathbb{Q}$, $\mathbb{C}$ und $\mathbb{R}$: \\
Damit ärgert man die Physiker! \\
Deshalb können wir existieren. \\
(3x) Es lebe Mathematik!}




