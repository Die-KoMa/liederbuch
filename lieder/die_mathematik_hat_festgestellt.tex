\lied{Die Mathematik hat festgestellt}{Die Wissenschaft hat festgestellt}{2005}

\strophe{ 
Die Mathematik hat festgestellt, \\
dass ne Menge sich nicht selbst enthält. \\
Drum beweisen wir auf jeder Reise, \\
Zorn'sches Lemma seitenweise.}

\strophe{
Der Cauchy, der hat schnell kapiert, \\
dass seine Folge konvergiert. \\
Und konstruiert auf diese Weise, \\ 
reelle Zahlen eimerweise.}

\strophe{
Die Algebra hat festgestellt, \\
dass kein Ideal die Eins enthält. \\
Und ist es auch noch maximal, \\
so ist's gleich ein Primideal.}

\strophe{
Den Griechen war es scheißegal, \\
sie dachten, Pi sei rational. \\
Doch heute weiß man ganz konkret, \\
dass $\mathbb{R}$ nicht nur aus $\mathbb{Q}$ besteht.}

\strophe{
Die Physiker haben festgestellt, \\
das Licht sich völlig falsch verhält. \\
Drum bestrahlen sie auf jeder Reise, \\
Doppelspalte teilchenweise.}

\strophe{
Der Einstein, der hat festgestellt, \\
dass Zeit sich relativ verhält. \\
Und darum gibt es keine Norm, \\
die herkommt von der Lorentzform.}

\strophe{
Die Didaktiker haben festgestellt, \\
dass ihr Studium Math'matik enthält. \\
Drum Addieren sie ganz still und leise, \\ 
Brüche Zähler- und Nennerweise. }

\strophe{
Die Numeriker haben ausprobiert, \\
dass $1$ durch $n$ doch konvergiert. \\
Drum machen sie auf diese Weise, \\
Rundungsfehler eps'lonweise.}

\strophe{
Die Statistiker haben festgestellt, \\
dass Zufall sich normal verhält. \\
Drum sperr'n sie ihn, das muss so sein, \\
unter Gauß'schen Kurven ein.}

\strophe{
Die KoMa, die hat festgestellt, \\
der Paulus, der hat sich verzählt. \\
Drum addieren wir konsequenterweise, \\
noch sechs hinzu zur nächsten Reise.}

\strophe{
Poincaré glaubte schon als Kind, \\
dass lochlos nur die Kugeln sind. \\
Doch leider war seit jener Zeit, \\
keiner zum Beweis bereit.}

\strophe{
Der Fermat, ja der hat gedacht, \\
er hätte den Beweis erbracht. \\
Doch leider fehlte ihm der Platz, \\
drum ist es nun der Wiles'sche Satz.}

\strophe{
Analysis hat festgestellt, \\
dass $\mathbb{R}^+$ Epsilon enthält. \\
Drum wählen wir von vornherein, \\
das Epsilon genügend klein.}




