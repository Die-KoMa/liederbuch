\lied{Hier kommt Euler}{Hier kommt Alex}{2007}

\strophe{
In dem $\mathbb{R}$-zwei, wo plötzlich nichts mehr lebt, \\
Wo jede Funktion nur noch nach unten geht, \\
Ist die größte Abwechslung, die es noch gibt, \\
Die allgegenwärt'ge Identität. \par
Sinus, Cos'nus wie ein Uhrwerk \\
Sind periodisch programmiert. \\
Es gibt keine Funktion, die noch nach oben strebt, \\
Auch alle Polynome sind frustriert!}

\refrain{
Hey, hey, hey, hier kommt Euler! \\
Vorhang auf für seine $e$-Funktion! \\
Hey, hey, hey, hier kommt Euler! \\
Vorhang auf für ein kleines bisschen $e$-Funktion!}
 
\strophe{
Auf dem Kreuzzug gegen die Ordnung \\
In dieser monotonen Welt \\
Strebt sie heftig gegen unendlich \\
So schnell, dass keiner sie hält. \par
Und wenn man sie jetzt diff'renzieren will, \\
Weil man denkt: Das bremst sie schon, \\
Hat man offensichtlich nichts kapiert: \\
$f$ Strich ist auch die $e$-Funktion!}

\strophe{(Refrain)}



