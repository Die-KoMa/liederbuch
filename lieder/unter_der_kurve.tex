\lied{Unter der Kurve}{Über den Wolken}{2005}

\strophe{ 
Ich hab hier ein Integral, \\
Und das macht mir große Sorgen. \\
Dacht' die Lösung wär trivial, \\
Doch jetzt bleibt sie mir verborgen. \\
Aber Riemann hat mir schon \\
Jenen winz'gen Tipp gegeben: \\
Approximiere monoton \\
Dem Limes entgegen.}

\refrain{
Unter der Kurve \\
Muss die Feinheit wohl grenzenlos sein. \\
Alle Folgen von Summen, sagt man, \\
Konvergieren dagegen und dann \\
Würde was uns noch unendlich erscheint, \\
Wie ein Epsilon klein.}

\strophe{ 
Doch jetzt kommt Aufgabe zwei, \\
ich muss wieder integrieren, \\
eine unstet'ge Funktion, \\
wie soll das nur funktionieren? \\
Aber diesmal gibt's Lebesgue, \\
der kann meine Nerven schonen, \\
approximiere einfach durch \\
Elementarfunktionen.}

\strophe{(Refrain)}

\strophe{
Doch bei der Dirichletfunktion \\
kann man damit nicht viel erreichen, \\
denn Lebesgue, der wusste schon, \\
das Riemann-Integral muss weichen, \\
für einen besseren Begriff \\
von Integralen und von Maßen, \\
einfach Funktionen war der Kniff, \\
die sich gut integrieren lassen.}



