\lied{Mein kleiner grüner Vektor}{"`Mein kleiner grüner Kaktus"' von Comedian Harmonists}{2008}

\strophe{
Die Vektor-Arten, die in Null starten, die nennt man auch die Ortsvektoren. \\
Ich hab' hier einen besonders kleinen, den habe ich mir auserkoren.}

\refrain{
Mein kleiner grüner Vektor steht ganz allein im Raum. \\
Holari, holari, holaro. \\
Er ist in Null gewurzelt, so fest, man glaubt es kaum. \\
Holari, \dotso \\
Er ist mir noch suspekt, so klein -- fast wie versteckt! \\
Drum nehm' ich 'nen Skalar, der ihn dann streckt, streckt, streckt. \\
Mein kleiner grüner Vektor ist plötzlich ziemlich groß. \\
Holari, \dotso}

\strophe{
Er ist nicht wendig und blickt beständig entlang der immergleichen G'raden. \\
Das find't er öde und denkt: \glqq Wie blöde! Ein Richtungswechsel würd' nicht schaden.\grqq }

\refrain{
Mein großer grüner Vektor, der hat 'nen Tunnelblick. \\
Holari, \dotso \\
Schaut ständig nur nach vorne und ab und an zurück. \\
Holari, \dotso \\
Dass es auch anders geht, ist klar -- wie ihr gleich seht: \\
Er sucht sich eine Matrix, die ihn dreht, dreht, dreht. \\
Mein großer grüner Vektor schaut jetzt woanders hin. \\
Holari, \dotso}

\strophe{
Zuviel Matrizen, er kommt ins Schwitzen -- so viele Winkel zum Probieren. \\
Er denkt: \glqq Wie schade, das ist doch fade, sich dreh'n und keinem imponieren.\grqq}

\refrain{
Mein großer grüner Vektor, der fühlt sich so allein. \\
Holari, \dotso \\
Er kann's nicht mehr ertragen und lädt zur Party ein. \\
Holari, \dotso \\
Die ganze Vektorschar ist da -- wie wunderbar! \\
Und keiner bleibt allein, das ist doch klar, klar, klar! \\
Mein großer grüner Vektor hat endlich, was er will. \\
Holari, \dotso}




