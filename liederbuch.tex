\documentclass[twoside]{report}

\usepackage{fixltx2e}
\usepackage[ngerman]{babel}
\usepackage[breakall,fit]{truncate} % Cut long headers
%\usepackage{color}
%\usepackage{calc}
\usepackage[utf8]{inputenc}
\usepackage{marvosym}
\usepackage[right]{eurosym}
\usepackage{amsmath}
\usepackage[absolute]{textpos}
%\usepackage{memhfixc}
\usepackage{wasysym}
\usepackage{eso-pic} % wasserzeichen
\usepackage{ellipsis}
\usepackage[pdftitle={Das Liederbuch der KoMa},pdfauthor={Konferenz der deutschsprachigen Mathematikfachschaften},final,pdfborder="0 0 0",pdfpagelayout=TwoColumnRight]{hyperref}
%\usepackage[squaren]{SIunits}
\usepackage{graphicx}
\usepackage{fullpagegraphic}
%\usepackage{epstopdf}
\usepackage{booktabs}
\usepackage{microtype}
\usepackage{multirow}
\usepackage[T1]{fontenc}
%\usepackage{blindtext}
\usepackage{setspace}
\usepackage{multicol}
\usepackage[labelformat=empty,justification=centering,textfont=normalsize]{subfig}
\usepackage{xspace}
\usepackage[a5paper,bindingoffset=10mm,twoside,vmargin=2.5cm,]{geometry}
\parindent=0pt

\usepackage{tocstyle} %%%
\usetocstyle{KOMAlike}
\settocfeature{leaders}{\hfill}% 
\setcounter{tocdepth}{2}
\setcounter{secnumdepth}{1}

\hypersetup{
colorlinks,%
citecolor=black,%
filecolor=black,%
linkcolor=black,%
urlcolor=black
} 

\usepackage{fancyhdr}
\pagestyle{fancy}
\fancyhead{} % Aufraeumen
\fancyhead[RO,LE]{\sffamily \small Seite \thepage}
\fancyhead[LO,RE]{\sffamily \small \truncate{0.6\textwidth}{\nouppercase\leftmark}}
\fancyfoot{} % Aufraeumen



\newcommand{\kifak}{\framebox{\tiny \textsc{KIF}}\mbox{\hspace{0.4cm}}}
\newcommand{\komaak}{\framebox{\tiny \textsc{KoMa}}\mbox{\hspace{0.4cm}}}

\newcommand{\supkif}{$^{\textsc{\tiny KIF}}$\xspace}
\newcommand{\supkoma}{$^{\textsc{\tiny KoMa}}$\xspace}



% Text muss auch mal GROSS sein!
\newcommand{\HUGE}{\fontsize{35.83}{43.00}\selectfont}
\newcommand{\titan}{\fontsize{43.00}{51.60}\selectfont}
\newcommand{\Titan}{\fontsize{51.60}{61.92}\selectfont}
\newcommand{\TITAN}{\fontsize{61.92}{74.30}\selectfont}
\newcommand{\TITELTEXT}{\fontsize{100}{100.30}\selectfont}
\newcommand{\ALL}[1]{\fontsize{#1}{1.2#1}\selectfont}


\usepackage{color}
\usepackage{amssymb}                 % ... und Symbole
\usepackage[T1]{fontenc}
\usepackage{lmodern}
%\usepackage{minitoc}                 % mehrere inhaltsverzeichnisse..
%\usepackage{ifthen}

\setcounter{secnumdepth}{-1}
%\setcounter{tocdepth}{1}     % tiefe des globalen toc
%\setcounter{minitocdepth}{3} % tiefe des lokalen toc

%%%%%%%%%%%%%%%%%%%%%%%%%%%%%%%%%%%%%%%%%

%\tolerance=2000
%\setlength{\emergencystretch}{20pt}

%\flushbottom            % für oneside: letzte Zeilen auf gleiche Höhe anpassen (bei twoside automatisch)

\setlength{\parindent}{0mm}
\setlength{\parskip}{1.5ex}

%\setlength{\hoffset}{-.5in}
%\setlength{\voffset}{-1cm}

%\setlength{\unitlength}{1cm}    %Einheitslänge = 1cm (für Bilder)

%\usetocstyle{standard}  %%%
%\newcommand\noop[1]{}   %%%
%\settocfeature[toc][0]{pagenumberbox}{\noop} %%%%

%%%% KOPF- & FUSSZEILEN
\fancypagestyle{plain}{ % "plain" damit es auch im TOC übernommen wird
  \fancyhead{} % Aufraeumen
  \fancyhead[ER]{\scshape \footnotesize \uppercase{Das Liederbuch}}
  \fancyhead[OL]{\scshape \footnotesize \uppercase{der KoMa}}
  \fancyhead[OR,EL]{\thepage}
  \fancyfoot{} % Aufraeumen
  \fancyfoot[OR]{\footnotesize http://www.die-koma.org}
  \fancyfoot[EL]{\footnotesize Stand: \Stand}
}
\fancypagestyle{pretoc}{
  \fancyhead{} % Aufraeumen
  \fancyhead[ER]{\scshape \footnotesize \uppercase{Das Liederbuch}}
  \fancyhead[OL]{\scshape \footnotesize \uppercase{der KoMa}}
  \fancyfoot{} % Aufraeumen
  \fancyfoot[OR]{\footnotesize http://www.die-koma.org}
  \fancyfoot[EL]{\footnotesize Stand: \Stand}
}

% Syntax: \lied{Liedtitel}
\newcommand{\lied}[3]{
\newpage
\section*{#1}
\addcontentsline{toc}{section}{#1\\(#2)}
{\vspace*{-2ex} \textit{Melodie: #2} \hspace*{1em} (#3) \vspace*{1ex} }
}

\newcommand{\strophe}[1]{
\begin{verse}
#1
\end{verse}
}

\newcommand{\refrain}[1]{
\begin{verse}
\textsl{Refrain:}\\
#1
\end{verse}
}


\title{Das Liederbuch der\\Konferenz der deutschsprachigen Mathematikfachschaften}
\author{Gedichtet vom AK Pella\\auf diversen KoMata}
\newcommand{\Stand}{14.~Nov 2022}
\date{Stand: \Stand}

\begin{document}
\pagenumbering{roman}
\pagestyle{empty}
\newpage
\includegraphicsfullpage{titelseite}
~
\vfill
\hfill
{\sffamily\LARGE\bfseries\textcolor{white}{\Stand}}
\hspace*{-15mm}
\vspace*{-14mm}

\pagestyle{pretoc}
\cleardoublepage
\pagestyle{plain}
\setcounter{page}{1}
%\addtocontents{toc}{\protect\thispagestyle{fancy}}

%\maketitle

\tableofcontents


\cleardoublepage
\pagenumbering{arabic}
\phantomsection
\addcontentsline{toc}{chapter}{Analysis}
\lied{Probier's doch mal mit Stetigkeit}{Probier's mal mit Gemütlichkeit}{2005}

\refrain{
Probier's doch mal mit Stetigkeit, \\
mit Eps'lon-Delta-Stetigkeit \\
wirfst du die blöden Folgen über Bord. \\
Und wenn du stets so stetig bist \\
und die Steigung etwas eklig ist, \\
dann diff'renzier sie dir doch einfach fort.}

\strophe{
Was soll ich woanders, wo's mir nicht gefällt,\\
ich gehe nicht fort hier, aus Bielefeld.\\
Der Kreidestaub zieht durch die Luft\\
erfüllt sie uns mit Matheduft.\\
Und schaust Du auf den Stein\\
siehst Du Variablen, die hier wohl gedeih'n.\\
Integrier' mal zwei, drei vier --\\
(Ist das dein Ernst? Hoho, es gibt nix besseres!)\\
denn mit Stetigkeit kommt auch das Glück zu Dir!\\
(Wie denn?)\\
Es kommt zu dir!}

\strophe{
(Refrain)}

\strophe{
Na, und differenzierst du gern Quotienten und irrst dich dabei,\\
dann lass dich belehren: Limites geh'n bald vorbei!\\
Du musst stetig und nicht sprunghaft im Leben sein, sonst tust du dir weh,\\
du bist verletzt und zahlst nur drauf,\\
drum differenziere gleich mit dem richt'gen Dreh!\\
Hast du das jetzt kapiert?\\
Denn mit Stetigkeit kommt auch das Glück zu dir!\\
Es kommt zu dir!}
\enlargethispage{4mm}
\strophe{
(Refrain)}

\lied{Unter der Kurve}{Über den Wolken}{2005}

\strophe{ 
Ich hab hier ein Integral, \\
Und das macht mir große Sorgen. \\
Dacht' die Lösung wär trivial, \\
Doch jetzt bleibt sie mir verborgen. \\
Aber Riemann hat mir schon \\
Jenen winz'gen Tipp gegeben: \\
Approximiere monoton \\
Dem Limes entgegen.}

\refrain{
Unter der Kurve \\
Muss die Feinheit wohl grenzenlos sein. \\
Alle Folgen von Summen, sagt man, \\
Konvergieren dagegen und dann \\
Würde was uns noch unendlich erscheint, \\
Wie ein Epsilon klein.}

\strophe{ 
Doch jetzt kommt Aufgabe zwei, \\
ich muss wieder integrieren, \\
eine unstet'ge Funktion, \\
wie soll das nur funktionieren? \\
Aber diesmal gibt's Lebesgue, \\
der kann meine Nerven schonen, \\
approximiere einfach durch \\
Elementarfunktionen.}

\strophe{(Refrain)}

\strophe{
Doch bei der Dirichletfunktion \\
kann man damit nicht viel erreichen, \\
denn Lebesgue, der wusste schon, \\
das Riemann-Integral muss weichen, \\
für einen besseren Begriff \\
von Integralen und von Maßen, \\
einfach Funktionen war der Kniff, \\
die sich gut integrieren lassen.}




\lied{Skandal im $\mathbb{R}$ hoch $n$}{Skandal im Sperrbezirk}{2006}

\strophe{ 
Im $\mathbb{R}$-zwo liegt der Einheitskreis, \\
Doch Einheitswurzeln müssen raus, \\
Damit auf diesem schönen Rand \\
Komplexes keine Chance hat! \par
Doch jeder hat schon längst gelernt, \\
Dass dies die Wahrheit ganz entfernt. \\
Es ist so schön und tut nicht weh: \\
$\mathbb{R}$-zwo ist isomorph zu $\mathbb{C}$!}

\refrain{
Und draußen vor dem Einheitskreis \\
Sind wir auf der Suche nach dem Beweis. \\
Skandal im $\mathbb{R}$ hoch $n$,  \\
Skandal im $\mathbb{R}$ hoch $n$, \\
Skandal! \\
Skandal um Cauchy!}

\strophe{ 
Der $\mathbb{R}$-$n$ war bekannt dem Gauß, \\
Doch Cauchy-Folgen müssen raus, \\
Damit in diesem großen Raum \\
Die Konvergenz stets bleibt ein Traum! \par
Sind Folgen auch nicht monoton, \\
Man findet oft ein Epsilon, \\
So dass von $a_n$ der Betrag \\
Sich jeder Grenze nähern mag.}

\strophe{(Refrain)}




\lied{Hier kommt Euler}{Hier kommt Alex}{2007}

\strophe{
In dem $\mathbb{R}$-zwei, wo plötzlich nichts mehr lebt, \\
Wo jede Funktion nur noch nach unten geht, \\
Ist die größte Abwechslung, die es noch gibt, \\
Die allgegenwärt'ge Identität. \par
Sinus, Cos'nus wie ein Uhrwerk \\
Sind periodisch programmiert. \\
Es gibt keine Funktion, die noch nach oben strebt, \\
Auch alle Polynome sind frustriert!}

\refrain{
Hey, hey, hey, hier kommt Euler! \\
Vorhang auf für seine $e$-Funktion! \\
Hey, hey, hey, hier kommt Euler! \\
Vorhang auf für ein kleines bisschen $e$-Funktion!}
 
\strophe{
Auf dem Kreuzzug gegen die Ordnung \\
In dieser monotonen Welt \\
Strebt sie heftig gegen unendlich \\
So schnell, dass keiner sie hält. \par
Und wenn man sie jetzt diff'renzieren will, \\
Weil man denkt: Das bremst sie schon, \\
Hat man offensichtlich nichts kapiert: \\
$f$ Strich ist auch die $e$-Funktion!}

\strophe{(Refrain)}




\lied{Transzendent}{Lady in black}{2008}

\strophe{
Drei Komma eins vier eins fünf neun \\
Zwei sechs fünf drei fünf acht neun sieben \\
Neun drei zwei drei acht vier sechs zwei \\
Sechs vier drei drei acht drei.}

\refrain{
Piiiiiiiii\dotso, Piiiiiiiii\dotso}
%3.14159 26535897 93238462 643383

\strophe{
Zwei Komma sieben eins acht zwei \\
Acht eins acht zwei acht vier fünf neun \\
Null vier fünf zwei drei fünf drei sechs \\
Null zwei acht sieben vier.}

\refrain{
Eeeeeeeeee\dotso, eeeeeeeeee\dotso}
%2.7182 81828459 04523536 02874

\strophe{
Eins Komma sechs eins acht null drei \\
Drei neun acht acht sieben vier neun \\
Acht neun vier acht vier acht zwei null \\
Vier fünf acht sechs acht drei.}

\refrain{
Phiiiiii\dotso, phiiiiiiii\dotso}
%1.61803 3988749 89484820 458683 43656




\lied{Epsilon}{Gummibären}{2009}

\strophe{
Winzig und niedlich, so lässig und friedlich, \\
immer beliebig, und doch größer Null. \\
Mit Delta zusammen, da macht es auch stetig, \\
in seiner Metrik, da ist es zu Haus.}

\refrain{
Epsilon, nutzt du hier und dort und überall, \\
es ist für dich da, wenn du es brauchst, \\
das ist das Epsilon.}

\strophe{
Die Griechen, die haben es für uns erfunden, \\
rund und geschwungen, so sieht es dann aus. \\
In Sätzen, Beweisen und Definitionen, \\
und auch im Beispiel, da wird es gebraucht.}

\refrain{
Epsilon, nutzt du hier und dort und überall, \\
es ist für dich da, wenn du es brauchst, \\
das ist das Epsilon.}

\strophe{
Das ist das Epsilon.}


\lied{Bin in $\mathbb{R}$}{"`Zombie"' von den Cranberries}{2010}

\strophe{
Nach all den Sattelpunkten, \\
nach den Miiiinima, \\
geht es dann nur noch höher, \\
weiter, weiter hinauf. \\
Ob in $\mathbb{R}$, ob in $\mathbb{Q}$, \\
Hauptsache nicht in $\mathbb{C}$ \\
Bin in $\mathbb{R}$, bin in $\mathbb{R}$, \\
komm hier nicht raus! }

\refrain{
Hilf mir schnell, hilf Newton \\
Hilf mir schnell! Hilf Newton! \\
Bin in $\mathbb{R}$, bin in $\mathbb{R}$, \\
ich divergier! \\
Kurz vorm Pol, kurz vorm Pol, \\
da wird's steil, da wird's steil, da wird's steil, steil, steil! \\
So kurz vorm Pol, kurz vorm Pol, \\
da wird's steil, da wird's steil, hoch hinaus-aus-aus-aus...}

\strophe{
In vielen andern Räumen \\
passiert so etwas nicht \\
da gibt's Schranken, Ränder, Kanten \\
alles bleibt so endlich. \\
Ist das selb-ee Spiel \\
Manche lösen es nie \\
wird zu viel, viel zu viel, \\
unendlich viel. }

\refrain{
Hilf mir schnell, hilf Banach \\
Hilf mir schnell! Hilf Banach! \\
Bin in $\mathbb{R}$, bin in $\mathbb{R}$, \\
ich divergier! \\
Kurz vorm Pol, kurz vorm Pol, \\
da wird's steil, da wird's steil, da wird's steil, steil, steil! \\
So kurz vorm Pol, kurz vorm Pol, \\
da wird's steil, da wird's steil, hoch hinaus-aus-aus-aus...}


\lied{Analysis kennt auch keine Lösung}{Kein Alkohol ist auch keine Lösung}{2011}

\strophe{
Es gibt Räume mit zu vielen Löchern,\\
Umgebung'n die sind stets zu groß.\\
Es gibt Terme, die bringen dir Schmerzen;\\
sind bei 0 wesentlich singulär.}

\strophe{
Doch meistens ist es wie immer:\\
Alles ist im \(\mathbb{C}^n\)\\
Und manchmal kommt es noch besser\\
da gilt selbst die Maximums-Norm.}

\strophe{
Was kann man auch sonst schon groß machen?\\
Was weiß man sonst über Funktion'?\\
Da kann man ja nichteinmal diff'renzieren\\
und Stetigkeit ist dann nur ein Traum.}

\strophe{
Das Differential ist des Rätsels Lösung\\
na klar, das kann schon sein.\\
Es gibt so viel schlaue Sätze dazu,\\
doch die helfen auch keinem Schwein!}

\refrain{
Analysis kennt auch keine Lösung\\
Ich hab es immer wieder versucht\\
Analysis kennt auch keine Lösung\\
Es geht lokal, doch das ist nicht g'nug.}

\strophe{
Manchmal will ich auch was integrieren\\
an der einen Kurve entlang\\
und ist sie auch noch so geschlossen\\
komm ich nicht bei 0 wieder an.}

\strophe{
Und dann frag ich mich wie auch schon früher:\\
wer kann denn bloß diesen Quatsch?\\
Ich hege Selbstzweifel ob meiner Mühe:\\
Ist denn Mathe nun doch nicht mein Fach?}

\strophe{(Refrain)}




\lied{Ahnungslos durch das Fach (Ana2)}{Atemlos durch die Nach}{2022}

\strophe{
Ich gehe in die Uni, hab' ne Prüfung vor mir.\\
Habe keine Ahnung, optimal wär ne Vier.\\
Ooh, ooh.\\
Ich schließe meine Augen bete schnell zu Fatou.\\
Maß und Integral gehören beide dazu
Ooh, ooh.
}

\strophe{
Was auf der Klausur auch steht,\\
Zeit, die viel zu schnell, vergeht.\\
Und ein Blick hat mir gezeigt:\\
Ich hab' es vergeigt.
}

\strophe{
Ahnungslos durch das Fach,\\
vielleicht reicht's mit Ach und Krach.\\
Ahnungslos, einfach raus,\\
die Versuche geh'n mir aus.
}

\strophe{
Ahnungslos durch das Fach,\\
Regelstudienzeit fällt flach.\\
Ahnungslos, Ana zwei,\\
Viertversuch, ich bin dabei.\\
}

\strophe{
Das wird heute eh nichts, wer soll das verstehen?\\
Alles was ich weiß, ist einfach fort.\\
Was bringt mir Fubini, wo ist hier Tonelli?\\
Welchen Satz benutze ich bloß dort?
}

\strophe{
Ich betrachte diese dumme Cantor-Funktion:\\
Wie ist die bloß stetig, ich verzweifel schon.\\
Ooh, ooh.\\
Ist das wirklich richtig, überhaupt konvergent?\\
Mir wird langsam klar: Ich bin voll inkompetent.\\
Ooh, ooh.
}

\strophe{
Alles, was ich brauch, unklar.\\
Die Verzweiflung ist ganz nah.\\
Ich will einfach nur noch weg,\\
bevor ich hier verreck'.
}

\strophe{
Ahnungslos durch das Fach,\\
Regelstudienzeit fällt flach.\\
Ahnungslos, Ana zwei,\\
Viertversuch, ich bin dabei.
}

\strophe{
Das wird heute eh nichts, wer soll das verstehen,\\
alles was ich weiß, ist einfach fort.\\
Was bringt mir Fubini, wo ist hier Tonelli?\\
Welchen Satz benutze ich bloß dort?
}

\strophe{
Ahnungslos\dots
}

\strophe{
Frust pulsiert auf meiner Haut.\\
Ahnungslos durch das Fach,\\
Regelstudienzeit fällt flach.\\
Ahnungslos, Ana zwei,\\
Viertversuch, ich bin dabei.
}

\strophe{
Das wird heute eh nichts, wer soll das verstehen,\\
alles was ich weiß, ist einfach fort.\\
Was bringt mir Fubini, wo ist hier Tonelli?\\
Welchen Satz benutze ich bloß dort?
}

\strophe{
Ahnungslos\dots
}



\cleardoublepage
\phantomsection
\addcontentsline{toc}{chapter}{Algebra, Topologie und Geometrie}
\lied{Die konvexe Hülle}{Die perfekte Welle}{2006}

\strophe{ 
Diese Hülle lag im Raum, \\
ein konvexes Polyeder. \\
Du bewunderst sie kaum, \\
weil du meinst, das kann ja jeder. \\
Hast schon stundenlang gerechnet, \\
hattest fast interpoliert, \\
doch die Strecke kommt nicht an, \\
sind die Punkte auch zu viert. \par
Jetzt kommt sie langsam auf dich zu, \\
du greifst schon mal zu deinem Stift, \\
malst die Verbindungsstrecke ein, \\
du kannst nicht glauben, dass sie trifft!}

\refrain{
Das ist die konvexe Hülle, \\
das ist jeder Zwischenpunkt. \\
Lass dich konvex kombinieren, \\
sind die Mengen auch disjunkt! \\
Das ist die konvexe Hülle, \\
das ist jeder Zwischenpunkt. \\
Es gibt mehr als du zählst, \\
es gibt mehr als du malst!}




\lied{Durch die Moduln}{Durch den Monsun}{2006}

\strophe{
Vektoren gibt es hier nicht mehr \\
Doch sie sind häufig unitär \\
Und deshalb rechnet man Strukturen aus \\
Ich grüble schon 'ne Ewigkeit \\
über die Invertierbarkeit \\
Und es gibt hier auch keinen Satz von Gauß.}
 
\refrain{
Ich muss durch die Moduln \\
über dem Ring \\
Solang' bis ich keine \\
Inversen mehr bring' \\
Mit Gruppenstruktur \\
An Skalaren entlang \\
Wenn ich nicht weiterkann, \\
dann schreib ich's dran. \\
Irgendwann rechnen wir zusammen \\
Durch die Moduln.}




\lied{Freiheit}{Freiheit}{2005}

\strophe{ 
Die Axiome sind gemacht, \\
auch an Ringe wurd' gedacht, \\
noch ein hübsches Korollar: \\
Freiheit, Freiheit. \\%(, die ist leider noch nicht da.)
Vieles wurde ausprobiert, \\
das Problem analysiert, \\
aber bald hat man kapiert:}

\refrain{
Freiheit, Freiheit ist das, was bei Moduln fehlt, \\
Freiheit, Freiheit, damit man die Basis zählt.}

\strophe{
Der Modul ist nicht projektiv, \\
der Körper, der ist leider schief. \\
Freiheit... \\
Alle, die von Freiheit träumen, \\
bleiben bei den Vektorräumen, \\
und sie pfeifen auf Torsionen.}

\refrain{
Freiheit, Freiheit, damit man die Basis zählt, \\
Freiheit, Freiheit ist das, was bei Moduln fehlt.}




\lied{Topologen auf dem Möbiusband}{Drei Chinesen mit dem Kontrabass}{2005}

\strophe{
Topologen auf dem Möbiusband, \\
Ohne Orientierung auf dem Weg zum Rand. \\
Doch Studenten ist es längst bekannt, \\
Dass die $S^1$ ist der Möbiusrand.}

\textit{Und nun alle Vokale durch einen ersetzen und nochmal singen.}




\lied{Wir rechnen modulo}{Pippi Langstrumpf}{2006}

\strophe{ 
Drei mal vier ist fünf \\
Widde-widde-witt und sechs macht viere. \\
Wir rechnen uns die Welt \\
Widde-widde-wie sie uns gefällt.}

\refrain{
Hey, wir rechnen modulo, \\
In diesem Fall mod sieben. \\
Hey, wir rechnen modulo, \\
Das ist, was uns gefällt!}

\strophe{ 
Drei hoch drei ist zwei \\
Widde-widde-witt und vier macht einen. \\
Wir definieren uns die Welt \\
Widde-widde-wie sie uns gefällt.}

\strophe{(Refrain \textit{(mod fünfe)})}

\strophe{ 
Das ist nicht schwer \\
Ja wirklich gar nicht schwer. \\
Das kenn ich von der Uhr, \\
Das nutz ich beim Kalender. \\
Das ist nicht schwer, \\
Das kann ja jedes Kind. \\
Und der, der sich nicht wehrt, \\
Kriegt unser Modulo gelehrt.}

\strophe{ 
Eins und zwei gibt nix \\
Widde-widde-wer will's von uns lernen. \\
Rechne ich mod drei \\
Gibbe-gibbe-gibt's nur null, eins, zwei.}

\strophe{(Refrain \textit{(mod drei)})}




\lied{Mein kleiner grüner Vektor}{"`Mein kleiner grüner Kaktus"' von Comedian Harmonists}{2008}

\strophe{
Die Vektor-Arten, die in Null starten, die nennt man auch die Ortsvektoren. \\
Ich hab' hier einen besonders kleinen, den habe ich mir auserkoren.}

\refrain{
Mein kleiner grüner Vektor steht ganz allein im Raum. \\
Holari, holari, holaro. \\
Er ist in Null gewurzelt, so fest, man glaubt es kaum. \\
Holari, \dotso \\
Er ist mir noch suspekt, so klein -- fast wie versteckt! \\
Drum nehm' ich 'nen Skalar, der ihn dann streckt, streckt, streckt. \\
Mein kleiner grüner Vektor ist plötzlich ziemlich groß. \\
Holari, \dotso}

\strophe{
Er ist nicht wendig und blickt beständig entlang der immergleichen G'raden. \\
Das find't er öde und denkt: \glqq Wie blöde! Ein Richtungswechsel würd' nicht schaden.\grqq }

\refrain{
Mein großer grüner Vektor, der hat 'nen Tunnelblick. \\
Holari, \dotso \\
Schaut ständig nur nach vorne und ab und an zurück. \\
Holari, \dotso \\
Dass es auch anders geht, ist klar -- wie ihr gleich seht: \\
Er sucht sich eine Matrix, die ihn dreht, dreht, dreht. \\
Mein großer grüner Vektor schaut jetzt woanders hin. \\
Holari, \dotso}

\strophe{
Zuviel Matrizen, er kommt ins Schwitzen -- so viele Winkel zum Probieren. \\
Er denkt: \glqq Wie schade, das ist doch fade, sich dreh'n und keinem imponieren.\grqq}

\refrain{
Mein großer grüner Vektor, der fühlt sich so allein. \\
Holari, \dotso \\
Er kann's nicht mehr ertragen und lädt zur Party ein. \\
Holari, \dotso \\
Die ganze Vektorschar ist da -- wie wunderbar! \\
Und keiner bleibt allein, das ist doch klar, klar, klar! \\
Mein großer grüner Vektor hat endlich, was er will. \\
Holari, \dotso}





\lied{Möbius Theme}{Darkwing Duck Theme}{2010}

\strophe{
Orientiern geht hier nicht\\
kompakt und chiral\\
einbettet in $\mathbb{R}^3$\\
Topologen sind dabei\\
Sphären, Tori, Zylinder\\
sind doch trivial!}

\refrain{
Zwei eins Möbius\\
Topologen wern zusammen gezogen\\
Möbius

Zwo Eins Epsilon\\
Möbius\\
August, Möbius.}

\strophe{
Singuläre Blätterung\\
Simplizes und mehr\\
berechne schnell den Linsenraum\\
da hilft sogar Baire\\
Hausdorff der separiert\\
macht die Basis abzählbar\\
Hier kommt:}

\refrain{
Möbius (genau?)\\
Topologen wern zusammen gezogen\\
Möbius

Zwo Eins Epsilon\\
Möbius\\
Passt bloß auf ihr Topologen

Möbius}



\cleardoublepage
\phantomsection
\addcontentsline{toc}{chapter}{Numerik und Stochastik}
\lied{Die Binomialverteilung Maja}{Die Biene Maja}{2005}

\strophe{
In einem unbekannten Raum \\
Vor gar nicht allzu langer Zeit, \\
als Diagramm in Form vom Baum \\
war die Wahrscheinlichkeit verteilt.}

\refrain{
Und diese Binomialverteilung, \\
Nennt sich Maja \\
Kleine, binomial verteilte Maja \\
Maja fliegt durch's Vektorfeld \\
Zeigt uns, wie ein Vektor fällt. \\
Wir treffen heute unsere binomiale Maja. \\
Diskrete Zufallsvariable Maja \\
Maja, so schön wie Fermat, ja, \\
Maja (Maja), Maja (Maja) \\
Maja verteile uns auf dir.}

\strophe{ 
Wenn ich an einem schönen Tag \\
Durch die Ereignismenge geh, \\
und dieses Wurfergebnis seh, \\
find ich ein W.-Maß, das ich mag.}

\strophe{
(Refrain)}

\strophe{
(Willy: Verteil' nicht so schnell, Maja.)}

\lied{Habe hier ein Problem}{Heute hier, morgen dort}{2007}

\strophe{ 
Habe hier oder dort ein Problem, das muss fort, \\
Hab schon vieles deswegen probiert. \\
Hab es differenziert und partiell integriert \\
Und sogar schon Fourier-transformiert.}

\refrain{
Manchmal ist es zu schwer, geht analytisch nichts mehr. \\
Dann gilt es für uns nun, das numerisch zu tun. \\
Diskretisiere, oh ja, und es ist uns ganz klar: \\
Der PC, der PC ist ja da!}

\strophe{
Fang mit dem Rechnen an, doch es dauert sehr lang, \\
Was ich noch nicht so wirklich begreif'. \\
Als ich analysier', ja da dämmert es mir, \\
Das Problem, das ist einfach zu steif.}

\strophe{(Refrain)}
 
\strophe{
Ich versuch' es nochmal, wie genau, ist egal, \\
Doch die Rechnung, die bricht leider ab. \\
Ich guck' mir das mal an, und es liegt wohl daran, \\
Dass ich dabei durch Null geteilt hab.}

\strophe{(Refrain)}

\strophe{
Ich hab' noch nicht genug, start' nen nächsten Versuch, \\
Und er geht sogar durch, der Anlauf. \\
Doch es geht nicht so recht, und die Lösung ist schlecht, \\
Denn es tritt dabei Auslöschung auf.}

\enlargethispage{3mm}
\strophe{(Refrain)}

\lied{Ich will rechnen}{I am sailing}{2008}

\strophe{
Ich will rechnen, habe Spaß dran, fühl' mich wichtig und auch schlau. \\
Warum rechnen, was bringt Rechnen, ist doch alles schon bekannt? \\
Will nicht rechnen, geht per Computer doch auch schneller als per Hand. \\
Warum rechnen, wer will schon rechnen, das ist völlig unnötig. \\
Will nicht rechnen, rechnen dauert und es bringt uns einfach nichts. \\
Ich muss rechnen, immer rechnen, für Klausuren, für den Prof. \\
Ich muss rechnen, sinnlos rechnen, Zahlen dreh'n sich in meinem Kopf.}




\lied{Numeriker}{Millionär}{1991}

\refrain{
Ich wär‘ so gern Numeriker,\\
dann wär‘ die Lösung nicht so schwer.\\
Ich wär‘ so gern Numeriker –\\
Numeriker.\\
Ich wär‘ so gern Numeriker.}

\strophe{
Term. Term. Term…\\
Ahhh…}

\strophe{
Hab‘ kein Beweis hab‘ keine Ahnung\\
und ich hab‘ keinen Plan.\\
Bin weder Doktor noch Professor –\\
dazu fehln‘ ein paar Verfahren.\\
Was sind Eigenwertprobleme?\\
Hab den Kurs dazu versäumt.\\
Von einer trivialen Lösung \\
hab ich leider nur geträumt.}

\strophe{
Was soll ich tun, was soll ich machen?\\
Vielleicht schreib ich ein Programm?\\
Habs mir ein paar mal überlegt,\\
doch ich hab zu wenig RAM.\\
Vielleicht frag ich nen Dozenten?\\
Allein komm ich nicht weit.\\
Der könnte meinen Stress beenden,\\
doch er hat keine Zeit.}

\strophe{
Ahhh…\\
(Refrain)}

\strophe{
Es gibt so viele große Firmen,\\
die begehrn mich sehr.\\
Sie sind so scharf auf meine Arbeit,\\
doch die ist mir zu schwer.\\
Ich glaub die würd ich nicht verkraften –\\
um keinen Preis der Welt.\\
Deswegen werd ich BWLer\\
und träume von viel Geld.}

\strophe{
Geld. Geld. Geld…\\
Ahhh…\\
(Refrain)\\
Ahhh…\\
(Refrain)\\
(Refrain)}


\cleardoublepage
\phantomsection
\addcontentsline{toc}{chapter}{Unileben}
\lied{Die Antwort weiß ganz allein der Prof}{Blowing in the Wind}{2006}

\strophe{ 
Wieviele Basen hat ein Vektorraum? \\
Die Frage ist äußerst beliebt. \\
Ja und wieviele Zykel sind in einem Baum? \\
Ist klar, dass es da keinen gibt? \\
In der Nacht überkommt mich der Prüfungsalptraum \\
Ob er den Termin noch verschiebt?}

\refrain{
Die Antwort mein Freund, weiß ganz allein der Prof. \\
Die Antwort weiß ganz allein der Prof.}
 
\strophe{
Wieviele Moduln sind heut' noch nicht frei \\
Und würden es so gerne sein? \\
Ja, was ist denn so schlimm an Pi gleich 3? \\
Der Fehler ist doch nur so klein. \\
Tja und jetzt ist die Prüfung nun endlich vorbei \\
Wofür nur erhalt ich den Schein?}

\strophe{(Refrain)}




\lied{Zehn kleine Erstsemester}{Zehn kleine Negerlein}{2005}

\strophe{
Über 10 kleine Erstsemester wollten wir uns freu'n, \\
einer wurde Physiker, da waren's nur noch neun.}

\strophe{
9 kleine Erstsemester fuhren durch die Nacht, \\
einer hatte's Ticket nicht, da waren's nur noch acht.}

\strophe{
8 kleine Erstemester waren eingeschrieben, \\
einer rafft' den Vorkurs nicht, da waren's nur noch sieben.}

\strophe{
7 kleine Erstsemester waren ganz perplex, \\
die OE war dann doch zuviel, da waren's nur noch sechs.}

\strophe{
6 kleine Erstsemester auf der Ersti-Fahrt, \\
der Alk, der floß in Strömen, das war dann doch zu hart.}

\strophe{
5 kleine Erstsemester die erste Woche hier, \\
der eine fand den Hörsaal nicht, da waren's nur noch vier.}

\strophe{
4 kleine Erstsemester, die brachten sich nichts mit, \\
sie gingen in die Mensa, da war'n sie noch zu dritt.}

\strophe{
3 kleine Erstsemester kamen auch nicht weit, \\
einer war in der Bib zu laut, da waren sie noch zu zweit.}

\strophe{
2 kleine Erstsemester mussten tapfer sein, \\
nur einer schaffte die Klausur, da war er ganz allein.}

\strophe{
1 kleiner Erstsemester konnt's schon nicht mehr seh'n, \\
drum wurd' er OE-Tutor, und jetzt gibt's neue zehn!}




\lied{Die Intermathionale}{Die Internationale}{2005}

\strophe{
Wacht auf Studenten dieser Uni, \\
die stets man noch zum Rechnen zwingt, \\
Der Bronstein und ein schneller Rechner \\
doch viel mehr Leistung bringt. \par
Lasst uns doch lieber was beweisen,  \\
der Computer rechnet's aus, \\
Wir können mehr als Arithmetik, \\
Adam Riese reicht nicht aus.}

\refrain{
Völker löst Integrale, auf zum nächsten dx. \\
die Raumdiagonale, die nützt uns hierbei nix.}

\refrain{
Es rettet uns kein höh'res Wesen, \\
kein Gauß, kein Cauchy, kein Laplace, \\
die Integrale hier zu lösen, \\
macht selbst den Profs kein' Spaß. \par
Nullmengen machen keinen Ärger, \\
Nullmengen lassen wir ganz weg. \\
Die Maßlosigkeit des Herrn Riemann \\
beseitigt der Lebesgue.}

\strophe{(Refrain)}

\strophe{
In Uni und FH ihr Studis, \\
sind wir die kleinste Fakultät, \\
die Ingenieure schiebt beiseite, \\
ohne uns hier nichts mehr geht. \par
Unser Werk sei nicht mehr \\
der Wiwis und der Biologen Fraß, \\
denn wenn sie sich verrechnet haben, \\
dann lachen wir uns was.}

\vspace*{-4mm}
\enlargethispage{8mm}
\strophe{(Refrain)}

\lied{Was sollen wir rechnen}{Was sollen wir trinken}{2006}

\strophe{ 
|: Was sollen wir rechnen, sieben Tage lang \\
Was sollen wir rechnen, oh mein Prof? :| \par
Mir fällt bestimmt noch schnell was ein \\
Ich lasse euch leiden, im dunklen Kämmerlein \\
Ich lasse euch leiden, das muss sein.}

\strophe{
|: Wir kommen nicht weiter, geb'n Sie uns nen Tipp \\
Wir kommen nicht weiter, so ein Mist :| \par
Das kann so schwer doch gar nicht sein! \\
Wer fleißig denkt, dem zeigt's sich von allein \\
Wer fleißig denkt, dem winkt der Schein.}

\strophe{
|: Wir haben ne Lösung, schau'n Sie das mal an \\
Wir haben ne Lösung, ist das toll! :| \par
Ich hab es euch doch gleich gesagt \\
Ich bin so stolz, denn ihr seid nicht verzagt \\
Ich bin so stolz, ihr seid begabt.}




\lied{Rechnen, rechnen, rechnen}{Saufen, saufen, saufen}{2007}

\strophe{ 
Es gibt Mengen, die sind mehr als mächtig. \\
Es gibt Mengen ohne was drin. \\
Es gibt Formeln ohne Symbole. \\
Es gibt Beweise ohne Sinn. \par
Es gibt Zahlen, die kann man nicht zählen. \\
Es gibt Leute, die teilen durch Null. \\
Es gibt schnelle Sekretärinnen \\
Und es gibt Tee, so gut wie Red Bull. \par
Es gibt schon so viel und es wird immer mehr, \\
Und wir können alles glauben. \\
Aber am schlimmsten ist immer noch: \\
Lösungen zu rauben.}

\refrain{
(Darum:) Rechnen, rechnen, rechnen, rechnen, \\
Rechnen, probieren, zerreißen. \\
Rechnen, rechnen, rechnen und am Ende \\
Doch noch beweisen.}
 
\strophe{
Es gibt Primzahlen, die sind grade. \\
Es gibt Sitze, die sind spitze. \\
Es gibt unentscheidbare Aussagen \\
Und unendlich schlechte Witze. \par
Es gibt Vorlesungen für Kinder. \\
Es gibt junge Professoren. \\
Es gibt Erstsemestereinführungen \\
Mit doll motivierten Tutoren. \par
Sie sagen, für Gebühren ist es nie zu spät. \\
Doch dabei verkennen sie die Realität: \\
Studenten auf der Straße fahnenschwenkend laufen. \\
Aber am besten ist immer noch: \\
Lösungen verkaufen.}

\strophe{(Refrain)}

\strophe{\textsl{Alternativer zweiter Refrain:} \\
Rechnen, rechnen, rechnen, rechnen, \\
Rechnen, streiken, agieren. \\
Rechnen, rechnen, rechnen und daneben \\
Noch protestieren.}




\lied{Verdammt, ich lös' das}{Verdammt, ich lieb dich}{2007}

\strophe{ 
Ich sitze hier am Schreibtisch bis nach Mitternacht. \\
Ich hab' das bisher schon oft gemacht. \\
Viel nützt es -- mir noch nicht. \par
Ich stehe an der Tafel, die ich hier vollschmier', \\
Doch die Lösung, sie verschließt sich mir, \\
Verschließt sich, verschließt sich mir. \par
Im Lehrbuch soll ein Ansatz sein, \\
Doch den sehe ich nicht ein. \\
Das ärgert, ärgert mich. \par
Auf einmal kommen mir neue Ideen, \\
Wär schön, wenn die jetzt endlich geh'n. \\
Doch es geht schon wieder nicht. \par
Und ich verzweifle schon wieder innerlich.}

\refrain{
Verdammt, ich lös das, ich lös das nicht! \\
Verdammt, ich schaff das, ich schaff das nicht! \\
%(Alternativ: Wie schafft Erdös das, ich schaff das nicht)
Verdammt, ich will das, ich will das nicht! \\
Ich will doch nicht frustrier'n!}
 
\strophe{
Langsam fällt mir doch 'ne Lösung ein: \\
Damit müsste das ganz einfach sein. \\
Bin fertig -- oder nicht? \par
Ist da was, das man vergisst? \\
Kann es sein, dass es jetzt richtig ist? \\
Ich glaub' das einfach nicht. \par
Gegenüber steht ein Telefon, \\
Ich ruf kurz an beim Komm'liton'. \\
Ich erreich' ihn, erreich' ihn nicht. \par
Der Kopf hat wieder mal zuviel geraucht, \\
Das ist es, was man in Mathe braucht, \\
Doch niemand, niemand sagt: Hör auf! \par
Und ich verzweifle schon wieder innerlich.}

\strophe{(Refrain)}





\lied{Unser Prof}{Meine Oma fährt im Hühnerstall Motorrad}{2006}

\strophe{ 
Unser Prof der prüft Numerik noch als Reine, \\
Unser Prof ist eine ganz gewitzte Sau!}

\strophe{ 
Sein Skript das ist von 1850, \\
Unser Prof ist eine ganz schön faule Sau!}

\strophe{ 
Das Zornsche Lemma das beweist er noch im Schlafe, \\
Unser Prof ist eine ganz gescheite Sau!}

\strophe{ 
Unser Prof der hat nen Freund der heißt Bernoulli, \\
Unser Prof ist eine ganz bekannte Sau!}

\strophe{ 
Unser Prof der schläft am liebsten gleich im Hörsaal, \\
Unser Prof ist eine ganz bequeme Sau!}

\strophe{ 
Unser Prof der ist allergisch gegen Kreide, \\
Unser Prof ist eine ganz schön arme Sau!}




\lied{Der Prof hat angefangen}{Der Mond ist aufgegangen}{2008}

\strophe{
Der Prof hat angefangen, die Studis müssen bangen. \\
Die Tafel leer und klar. \\
Jetzt holt er neue Kreide und blättert um die Seite \\
Des Skripts zur Lin'aren Algebra.}

\strophe{
Zum Anfang Def'nitionen, die sich auch alle lohnen \\
Im weiteren Verlauf. \\
Es folgen Theoreme, ganz schrecklich unbequeme. \\
Und neue Grenzen tun sich auf.}

\strophe{
Jetzt schnell noch die Beweise, und weiter geht die Reise \\
Ins große Mathe-Land. \\
Sie raufen sich die Haare, zum Lernen braucht man Jahre. \\
Und schon erscheint die nächste Wand.}

\strophe{
Der Frust wird immer größer, die Studis werden böser -- \\
Der Prof bleibt ungerührt. \\
Die Hörer rebellieren, woll'n aus dem Buch kopieren: \\
Wie wird denn der Beweis geführt?}




\lied{Fachschaftsraum}{Westerland}{2008}

\strophe{
Jeden Tag sitz' ich im Hörsaal, und ich hör' Dozenten zu. \\
Sie woll'n mich so vieles lehren, doch ich sehe nie den Clou. \\
Diese eine Stunde wird nie zu Ende geh'n, \\
Denn die Zeit scheint still zu steh'n. \par
Manchmal starr' ich auf den Zettel, davon kommt die Lösung nicht. \\
Dann schau ich nochmal zur Tafel -- Fragezeichen im Gesicht. \\
Diese eine übung wird nie zu Ende geh'n. \\
Wann wird da die Lösung steh'n?}

\refrain{
Oh, mir ist ja so öde! \\
Ich verfolg' die übung kaum. \\
Ich muss dringend Karten spielen -- \\
Ich will zurück zum Fachschaftsraum.}

\strophe{
Wie oft saß ich in der Prüfung, und ich dachte mir: Ohjeee\dotso \\
Keiner konnte mich noch retten, und ich wusste nichts von $\mathbb{C}$. \\
Diese Nachklausuren werd' ich wohl nie besteh'n. \\
Wann darf ich denn endlich geh'n?}

\refrain{
Oh, ich hab' keine Ahnung! \\
Das ist wie ein schlechter Traum. \\
Ich muss doch noch mehr studieren. \\
Ich will zurück zum Fachschaftsraum.}

\strophe{
Es ist zwar Zeitverschwendung, doch dort ist man unter sich. \\
Und ich weiß: Jeder Zweite hier braucht genauso lang wie ich.}

\refrain{
Oh, ich kann's nicht bezahlen! Soviel Geld, das schaff' ich kaum. \\
Meine Studiengebühren zahlt niemand aus dem Fachschaftsraum.}





\lied{Was machen wir mit den Erstsemestern}{What shall we do with the drunken sailor}{2008}

\textit{Es erwies sich als unpraktikabel, alle Strophen vorzutragen\dotso}

\strophe{
Was machen wir mit den Erstsemestern \\
Was machen wir mit den Erstsemestern \\
Was machen wir mit den Erstsemestern \\
In der Erstiwoche?}

\refrain{
$|\!:$ Hooray, wir sind Tutoren $:\!|$ (3x) \\
Für die Erstiwoche!}

\strophe{
$|\!:$ Wir begrüßen sie im Hörsaal $:\!|$ (3x) \\
In der Erstiwoche.}

\strophe{
Sie kriegen von uns 'ne Erstizeitung\dotso}

\strophe{
Wir teilen sie in kleine Gruppen\dotso}

\strophe{
Machen mit ihnen Kenn'lernspiele\dotso}

\strophe{
Wir beraten sie zu ihrem Studium\dotso}

\strophe{
Dann schicken wir sie auf die Rallye\dotso}

\strophe{
Jeder kriegt 'ne Kennung vom Rechenzentrum\dotso}

\strophe{
Essen mit ihnen in der Mensa\dotso}

\strophe{
Halten 'ne Vorlesung nur zur Probe\dotso}

\strophe{
Dann simulier'n wir 'ne Übungsgruppe\dotso}

\strophe{
Dann lernen sie die Fachschaft kennen\dotso}

\strophe{
Dann gibt es einen Spieleabend\dotso}

\strophe{
Wir spielen eine Runde Werwolf\dotso}

\strophe{
Wir zeigen ihnen alle Kneipen\dotso}

\strophe{
Feiern zusammen eine Party\dotso}





\lied{Fachschaftsleut'}{Hänschen klein}{2008}

\strophe{
Fachschaftsleut' gehen heut' \\
zu verschied'nem Mathezeug. \\
Viele dort an dem Ort -- \\
Was geht da nur fort? \\
Vierundzwanzig Stunden lang \\
Stell'n sie sich dem Mathedrang. \\
Vom Graph planar bis RSA -- \\
Dies ist unser Jahr!}





\lied{Suizidale Gedankengänge in Pi-moll}{Eternal Flame}{2008}

\strophe{
Klos im Hals, heut' ist Klausur -- Scheiße! \\
Und ich hab' sie vergessen! Und auch nicht gelernt. \\
Werd' ich untergehen? Exmatrikuliert? \\
Heut' ist Prüfung -- ich kann's nicht verstehen.}

\strophe{
Ich geh' hin zum Hörsaal 4, wieder. \\
Der Prof ist heut' wieder bieder, lächelt mich nur an. \\
Heute bist du dran! Und er wird noch fieser: \\
Er setzt sich vor mich -- abschreiben ist jetzt nicht!}

\strophe{
Suizid! Wäre eine Lösung, \\
Doch ich weiß: So richtig wär' das auch wieder nicht. \\
Denn dann komm' ich in die Hölle! Oh nein\dots}

\strophe{
Ich sitz' jetzt, schwitzend vor Panik, vor dem Prof. \\
Negative Gedanken schwirren um mich rum. \\
Und am Höhepunkt werd' ich wach -- ich war am Schlafen! \\
Ich lieg' im Schlafraum, das war alles bloß ein Traum!}





\lied{Tausend und ein Beweis}{Tausend und eine Nacht}{2008}

\strophe{
Ich wollte doch bloß meine Arbeit beenden, \\
Die letzten paar Seiten, dann hätt' ich Diplom. \\
Hab' nie viel verstanden von dem, was ich schreibe. \\
Zum Ziel kam ich trotzdem, was macht das dann schon? \par
Die Ander'n war'n besser, in vielen Belangen. \\
Nur mir fehlte immer der tiefere Blick. \\
Bis gestern am Schreibtisch, ich war wie gefesselt: \\
War das 'ne Erleuchtung -- oder war es nur Glück?}

\refrain{
Tausendmal studiert, \\
Tausendmal schon ausprobiert. \\
Tausend und ein Beweis -- \\
Und ich kapier' den Scheiß!}

\strophe{
Ganz plötzlich ist alles so vollkommen anders! \\
Auf einmal ergeben alle Sätze 'nen Sinn. \\
Ich sehe jetzt Brücken zu ander'n Gebieten \\
Und weiß, bis zum Ziel ist es nicht mehr weit hin. \par
Warum nicht schon früher so eine Erleuchtung? \\
Soviel Euphorie hätt' ich gern damals verspürt. \\
Am Anfang des Studiums war alles so mühsam. \\
Ich fühlte mich falsch hier und war so frustriert.}

\strophe{(Refrain)}

\strophe{
Doch das ist vergangen, ich schaue nach vorne. \\
Die neuen Gedanken bringe ich zu Papier. \\
Ich schreib' Theoreme, die nie wer erdachte. \\
Mein Prof wird gut staunen, wenn ich sie präsentier'. \par
Sogar das Beweisen geht plötzlich ganz einfach. \\
Das q.e.d. schreib' ich nach einer Stunde schon. \\
Die Arbeit ist fertig, ich kann es kaum fassen: \\
Ich bin Mathemat'ker -- ich hab' mein Diplom! \par
Ich hab' mein Diplom!}

\strophe{(Refrain)}



\lied{Wo ist der Scheiß-Beweis?}{Die Affen rasen durch den Wald}{?2009}

\strophe{
Die Studis sitzen vor dem Blatt, \\
Man kriegt's nicht raus, das macht sie platt. \\
Die ganze Übungsgruppe brüllt: }

\refrain{
Wo ist der Scheiß-Beweis? Wo ist der Schei\ss-Beweis? \\
Wer hat den Scheiß-Beweis geklaut? \\
Wo ist der Scheiß-Beweis? Wo ist der Schei\ss-Beweis? \\
Wer hat den Scheiß-Beweis geklaut?}

\strophe{
Im Skript steht auch nichts Gutes drin, \\
da müssen wir woanders hin. \\
Die ganze Übungsgruppe brüllt:}

\strophe{(Refrain)}

\strophe{
Der Tutor kommt in's Zimmer rein, \\
Auch ihm fällt keine Lösung ein. \\
Die ganze Übungsgruppe brüllt:}

\strophe{(Refrain)}

\strophe{
Dem Doktorand ist das egal, \\
Er nennt die Lösung trivial. \\
Die ganze Übungsgruppe brüllt:}

\strophe{(Refrain)}

\strophe{
Im ganzen Institut gibt's Zoff, \\
Die Lösung hat nicht mal der Prof. \\
Die ganze Übungsgruppe brüllt:}

\strophe{(Refrain)}

\strophe{
Der Autor hatte es kapiert, \\
Doch der ist längst emeritiert. \\
Die ganze Übungsgruppe brüllt:}

\strophe{(Refrain)}

\strophe{
Im alten Skript, da könnt' es sein, \\
Doch leider war der Rand zu klein. \\
Die ganze Übungsgruppe brüllt:}

\strophe{(Refrain)}

\strophe{
Die Sekretärin lacht sich schief, \\
Die Lösung lagert im Archiv. \\
Die ganze Übungsgruppe brüllt:}

\strophe{(Refrain)}

\strophe{	
Die Übungsgruppe schreit: "`Hurra!"', \\
Jetzt ist die Lösung allen klar! \\
(Die ganze Übungsgruppe brüllt:)}

\strophe{Refrain II: \\
Da ist der Scheiß-Beweis! Da ist der Schei\ss-Beweis! \\
Sie hat den Scheiß-Beweis geklaut! \\
Da ist der Scheiß-Beweis! Da ist der Schei\ss-Beweis! \\
Wer hat den Scheiß-Beweis geklaut!}

\strophe{
Und die Moral von der Geschicht': \\
Beweise archiviert man nicht! \\
Weil sonst die Übungsgruppe brüllt:}

\strophe{(Refrain)}


\lied{Schlaf, Ersti, schlaf}{Schlaf, Kindlein, schlaf}{2004}


\strophe{
Schlaf, Ersti, schlaf.\\
Der Assi\footnote{Abkürzung für "`(Wissenschaftlicher) Assistent"'} ist ein Schaf.\\
Der Prof beweist ins Blaue rein\\
und du bist jetzt das arme Schwein!\\
Schlaf, Ersti, schlaf.}


\cleardoublepage
\phantomsection
\addcontentsline{toc}{chapter}{Ringvorlesung}
\lied{Äquivalent}{Ein Loch ist im Eimer}{2005}

\strophe{
Wie beweis' ich das Lemma, das Lemma, das Lemma, \\
Wie beweis' ich das Lemma, das Lemma von Zorn?}

\strophe{
So nimm doch zu Hilfe, zu Hilfe, zu Hilfe, \\
So nimm doch zu Hilfe das Auswahlaxiom!}

\strophe{
Wer hat's mir gegeben, gegeben, gegeben, \\
Wer hat's mir gegeben das Auswahlaxiom?}

\strophe{
Zermelo und Fraenkel, und Fraenkel, und Fraenkel, \\
Zermelo und Fraenkel mit dem Wohlordnungssatz!}

\strophe{
Woher kommt er denn wirklich, denn wirklich, denn wirklich, \\
Woher kommt er denn wirklich, der Wohlordnungssatz?}

\strophe{
Der folgt aus dem Lemma, dem Lemma, dem Lemma, \\
Der folgt aus dem Lemma, dem Lemma von Zorn!}

\strophe{
\textit{Und nun von vorn! Und dabei immer schneller werden, ad infinitum...}}

\lied{Die Mathematik hat festgestellt}{Die Wissenschaft hat festgestellt}{2005}

\strophe{ 
Die Mathematik hat festgestellt, \\
dass ne Menge sich nicht selbst enthält. \\
Drum beweisen wir auf jeder Reise, \\
Zorn'sches Lemma seitenweise.}

\strophe{
Der Cauchy, der hat schnell kapiert, \\
dass seine Folge konvergiert. \\
Und konstruiert auf diese Weise, \\ 
reelle Zahlen eimerweise.}

\strophe{
Die Algebra hat festgestellt, \\
dass kein Ideal die Eins enthält. \\
Und ist es auch noch maximal, \\
so ist's gleich ein Primideal.}

\strophe{
Den Griechen war es scheißegal, \\
sie dachten, Pi sei rational. \\
Doch heute weiß man ganz konkret, \\
dass $\mathbb{R}$ nicht nur aus $\mathbb{Q}$ besteht.}

\strophe{
Die Physiker haben festgestellt, \\
das Licht sich völlig falsch verhält. \\
Drum bestrahlen sie auf jeder Reise, \\
Doppelspalte teilchenweise.}

\strophe{
Der Einstein, der hat festgestellt, \\
dass Zeit sich relativ verhält. \\
Und darum gibt es keine Norm, \\
die herkommt von der Lorentzform.}

\strophe{
Die Didaktiker haben festgestellt, \\
dass ihr Studium Math'matik enthält. \\
Drum Addieren sie ganz still und leise, \\ 
Brüche Zähler- und Nennerweise. }

\strophe{
Die Numeriker haben ausprobiert, \\
dass $1$ durch $n$ doch konvergiert. \\
Drum machen sie auf diese Weise, \\
Rundungsfehler eps'lonweise.}

\strophe{
Die Statistiker haben festgestellt, \\
dass Zufall sich normal verhält. \\
Drum sperr'n sie ihn, das muss so sein, \\
unter Gauß'schen Kurven ein.}

\strophe{
Die KoMa, die hat festgestellt, \\
der Paulus, der hat sich verzählt. \\
Drum addieren wir konsequenterweise, \\
noch sechs hinzu zur nächsten Reise.}

\strophe{
Poincaré glaubte schon als Kind, \\
dass lochlos nur die Kugeln sind. \\
Doch leider war seit jener Zeit, \\
keiner zum Beweis bereit.}

\strophe{
Der Fermat, ja der hat gedacht, \\
er hätte den Beweis erbracht. \\
Doch leider fehlte ihm der Platz, \\
drum ist es nun der Wiles'sche Satz.}

\strophe{
Analysis hat festgestellt, \\
dass $\mathbb{R}^+$ Epsilon enthält. \\
Drum wählen wir von vornherein, \\
das Epsilon genügend klein.}





\lied{Wir haben's nicht bewiesen}{"`Wir haben Grund zum Feiern"' von Otto}{2007}

\strophe{ 
Analysis und Algebra \\
Vektorpunkt und Vektorschar, \\
Matrixfeld und Skalar \\
sind doch wunderbar. \par
Koeffizient Binomial \\
Folge, Reihe, Integral, \\
Ableitung Differential, \\
$e$ hoch $x$, trivial. \par
Kleiner, größer, Inklusion,  \\
leere Menge, Projektion  \\
transitive Relation,  \\
ausgeartete Funktion.}

\refrain{
Wir haben's nicht bewiesen. \\
Doch wir wollen's nutzen, \\
auch wenn and're stutzen. \\
Wir haben's nicht bewiesen \\
Doch wir woll'ns riskieren \\
und den Kreis quadrieren.}

\strophe{
Polynome Eisensteins, \\
Kommutator, Ring mit Eins. \\
Primideale? Gibt es keins. \\
$i$ Quadrat ist minus Eins.}

\strophe{
Stirling-Formel, Fakultät \\
Kehrwert, Reziprozität. \\
Wenn nicht Null im Nenner steht, \\
teilen immer geht. \par
Netzwerkgraphentheorie, \\
Flüsse strömen wie noch nie, \\
Außenwinkel, Sinus phi, \\
Trigonometrie. \par
Null durch null nicht definiert, \\
Hat das irgendwer kapiert? \\
Pi im Kreis approximiert. \\
Mathe haben wir studiert.}

\strophe{(Refrain)}

\lied{Meine Art, Sätze zu zeigen}{Meine Art, Liebe zu zeigen}{2007}

\refrain{
Meine Art, Sätze zu zeigen, ist ein formaler Reigen. \\
Worte verstören, wo sie nicht hingehören. \\
Meine Art, Sätze zu zeigen, ist ein formaler Reigen. \\
Symbole klären, wo Worte sinnlos wären.}

\strophe{
Sieh die beiden Terme steh'n, sind sie gleich, hm, kann das geh'n? \\
auf die normale Form getrimmt, zeigt sich das bestimmt. \\
Skizziere oder mal' ein Bild, forme um und rechne wild, \\
setzt du Terme richtig ein, löst sich alles fein.}

\strophe{(Refrain)}
 
\strophe{
Schaue, was gegeben ist, liste die Prämissen \\
was gilt dann, das sicher ist, nach bekanntem Wissen? \\
Halt es fest, doch halt nicht an, folgere nur weiter, dann \\
stellet sich am Schluss allein, die Behauptung ein.}

\strophe{(Refrain)}

\strophe{
Weißt du, ist die Kurve glatt? Wo sie ihre Pole hat? \\
Hast du sie approximiert, schaue was passiert. \\
Zum Beweis der Stetigkeit, stell ein Epsilon bereit. \\
Wähl das Delta richtig klein, dann passt es hinein.}

\strophe{(Refrain)}





\lied{Summer Dreamin'}{Summer Dreamin'}{2007}

\strophe{
Come on over, have some fun, \\
Mathe in the morning sun. \\
Nimm mal hier das Möbiusband, \\
Komm mit mir und such den Rand. \par
Schau mal da, ein K-drei-drei. \\
Teil das Ding doch mal entzwei. \\
Kuratowski sagt uns klar: \\
Der Graph ist nicht plättbar.}

\refrain{
Diff'ren-zie-ren -- \\
It's never been so easy. \\
Inte-grie-ren -- \\
\textbf{(Hammerzeile fehlt)}  \\
(Original: Summer dreamin when you're with me.  \\
\textbf{Vorschläge:} \\
Leicht zu lösen when you're with me. \\
Easy going - Riemann with me!)}
 
\strophe{
Alpha, Beta, Omega. \\
Und das Tau ist auch schon da. \\
Hörst du nicht, sie rufen schon? \\
Ohne sie wär'n wir verlor'n. \par
Folge deinem inn'ren Drang. \\
Moduln machen uns nicht bang. \\
W-Maß, Sigma-Algebra -- \\
Doch ich mache lieber...}

\strophe{(Refrain)}

\lied{Zahlen}{Männer}{2008}

\strophe{
Zahlen machen uns arm, Zahl'n beschäftigen die Welt. \\
Zahlen können komplex sein, Zahlen können verwirrend sein. \\
Oh Zahlen sind auch gefährlich. \\
Doch Zahlen meinen es meistens ehrlich.}

\strophe{
Mit Zahlen kauft man ein, nur mit Zahlen geht alles gut. \\
Zahlen sind ziemlich öde, nur wenige ham Mut \\
Uns're Zahlen zu erfassen. \\
Auf diese Menschen hoch die Tassen!}

\refrain{
Manche Zahlen sind prim, manche ganz. \\
Zahl'n verleihen uns uns'ren Glanz. \\
Doch eins versteh' ich bis heute nicht: \\
(2x) Warum ist Pi reell? \\
Warum ist Pi nicht drei?}

\strophe{
Wenn Pi nur in $\mathbb{Z}$ wär', wäre die Welt ganz leicht. \\
Kreise hätten jetzt Ecken, hätte das nicht ausgereicht? \\
Warum muss man denn so genau sein? \\
Ein Kassenbon reicht heut' doch auch als Fahrschein!}

\strophe{(Refrain)}

\strophe{
Ganze Zahlen sind doch einfacher zu bestimm' \\
Und positiv helfen sie zu gewinn' \\
Gegen Bruchzahlen, gegen Kommata -- macht das einen Sinn?}

\refrain{
(2x) Warum ist Pi reell? \\
Warum ist Pi nicht drei?}

\strophe{
Was wäre Mathe eigentlich ohne Pi? \\
Jeder könnte rechnen, Physiker freuen sich wie nie! \\
Das könnt' man doch nicht machen -- \\
Physiker haben nichts zu lachen!}

\refrain{
Es gibt nen Grund für $\mathbb{Q}$, $\mathbb{C}$ und $\mathbb{R}$: \\
Damit ärgert man die Physiker! \\
Deshalb können wir existieren. \\
(3x) Es lebe Mathematik!}





\lied{53, 73, 19, 2003}{54, 74, 90, 2010}{???? / 2015}

\refrain{
  53, 73, 19, 2003,\\
  ja so stimmen wir alle ein!\\
  Mit dem Stift in der Hand\\
  und durch Teilen ohne Rest\\
  werden wir Primzahlen sein!
}

\strophe{
  Wir haben nicht\\
  die höchste Ringstruktur,\\
  haben nur den kleinen Fermat,\\
  doch wir haben Eulersche Funktionen,\\
  sind der Schlüssel zum RSA!\\
  Es sind nicht viele Faktoren\\
  Die Basis unseres Seins.\\
  Es lautet die Devise:\\
  Nur wenn man teilt, dann wird man Eins!
}

\strophe{(Refrain)}

\lied{Mathe}{Moskau}{2022}

\strophe{
Mathe\\
Sätze und Lemmata\\
vieles ist gar nicht klar\\
bis zum Beweis\\
Mathe\\
griechische Buchstaben\\
in uns'ren Hörsälen\\
ler'n wir voll Fleiß
}

\strophe{
Studenten hey, hey, hey, fahrt zur Uni\\
Vorlesung ha, ha, ha, das wird schön\\
Dozenten hey, hey, hey, an die Tafeln\\
Zeig den Weg Euler hey, Euler ho!
}

\strophe{
Mathe, Mathe, schreib die Formeln an die Wand\\
Kreidestaub auf unsrer Hand, ha ha ha ha ha, hey\\
Mathe, Mathe, deine Vielfalt ist so groß\\
vieles zeigt man rigoros, ha ha ha ha ha, hey\\
Mathe, Mathe, nicht mehr rechnen wollen wir,\\
wir sind zum Beweisen hier, ho ho ho ho ho, hey\\
Mathe, Mathe, Witze über Physiker\\
Biologen tun sich schwer, ha ha ha ha ha
}

\strophe{
Mathe\\
Tor zur Vergangenheit,\\
Spiegel von Riemanns Zeit\\
Cantor tut gut\\
Mathe\\
Wer deine Seele kennt\\
der weiß die Liebe brennt\\
heiß wie die Glut
}

\strophe{
Studenten hey, hey, hey, fahrt zur Uni\\
Vorlesung ha, ha, ha, das wird schön\\
Dozenten hey, hey, hey, an die Tafeln\\
Zeig den Weg Cauchy hey, Cauchy ho!
}

\strophe{
Mathe, Mathe, schreib die Formeln an die Wand\\
Kreidestaub auf deiner Hand, ho ho ho ho ho, hey\\
Mathe, Mathe, deine Vielfalt ist so groß\\
vieles zeigt man rigoros, ha ha ha ha ha, hey\\
Mathe Lala lala lala la, lala lala lala la\\
Ho ho ho ho ho, hey\\
Mathe Lala lala lala la, lala lala lala la\\
Ha ha ha ha ha\\
Oh, oh oh oh oh, oh oh oh oh, oh oh oh
}

\lied{Koordinaten}{Thüringer Klöße}{2022}

\strophe{
Fragt man mich, was machst du gern, da kann ich Antwort geben:\\
Koordinaten transformier'n, nur das brauch' ich zum Leben.\\
Erzählt der Prof' von PDEs, ist mir das ganz egal,\\
doch wenn ich Rotationen seh', so werd' ich hart wie Stahl.
}

\strophe{
Koordinaten, die mag ich sehr\\
die finde ich am besten.\\
Skalierung, Scherung, Translation\\
affine Trafos muss ich testen.\\
Ich bin danach ganz süchtig,\\
die Umformung ist wichtig.\\
Nicht Pfeile, nicht Vektoren, nein:\\
Koordinaten müssen's sein!
}

\strophe{
Haben wir Produkträume, so rechne ich kartesisch,\\
doch ist das Ganze ziemlich rund, so mach ich's lieber sphärisch\\
Natürlich hab' ich Gauß und Stokes vor langem schon studiert,\\
und alles von den Physikern ganz kritisch transformiert.
}

\strophe{
Koordinaten, die mag ich sehr\\
die finde ich am besten.\\
Skalierung, Scherung, Translation\\
affine Trafos muss ich testen.\\
Ich bin danach ganz süchtig,\\
die Umformung ist wichtig.\\
Nicht Pfeile, nicht Vektoren, nein:\\
Koordinaten müssen's sein!
}

\strophe{
Koordinaten, die mag ich sehr\\
die finde ich am besten.\\
Skalierung, Scherung, Translation\\
affine Trafos muss ich testen.\\
Ich bin danach ganz süchtig,\\
die Umformung ist wichtig.\\
Nicht Pfeile, nicht Vektoren, nein:\\
Koordinaten müssen's sein!\\
Koordinaten müssen's sein!
}



\cleardoublepage
\phantomsection
\addcontentsline{toc}{chapter}{KoMa}
\lied{Die Reso vor mir}{Im Wagen vor mir}{1987}

\strophe{
An der Wand vor mir hängt 'ne neue Reso,\\
Ich kenn sie nicht, doch sie scheint doof zu sein.\\
Ich weiß nur ihren Namen, doch kenne nicht ihr Ziel,\\
ich merke nur sie bringt uns hier nicht viel.
}

\strophe{
Die Reso vor mir ist von einem Dulli,\\
Ich möcht' gern wissen, was der sich dabei denkt;\\
Hört der seinen Müll nicht oder ist sein Hirn schon aus;\\
Möcht' er uns ärgern, ich schleif ihn gleich hier raus!
}

\refrain{
Rada rada radadadada, rada rada radadadada
}

\strophe{
Was will das blöde Plenum da von mir nur? (Ist das ein Müll!)\\
Ich frag' mich, warum veto'n die denn bloß? (Dieser Satzbau!)\\
Wir diskutieren schon zwei Stunden ständig hin und her,\\
nun dämmert schon und im Saal gibt es kein Licht. (Erstmal zum Frühstück.)\\
Ich könnt schon 100 km weg sein. (Ich muss aufs Klo.)
}

\strophe{
(3 mal Refrain)
}
\lied{KoMa-Land}{Eine Insel mit zwei Bergen}{2011}

\strophe{
Eine KoMa mit Studenten,\\
auf dem Plenum in der Nacht,\\
diskutieren wild und eifrig,\\
über was sie hier verzapft.}

\strophe{Viele AKs sind gelaufen\\
viele Brötchen sind verzehrt\\
gab es nicht genug zu essen\\
ja dann haben wir uns beschwert.
}

\refrain{(Pfeifen)}

\strophe{
Mit dem Aufstehen gibt's Probleme\\
waren wach die ganze Nacht\\
denn sie saßen mit dem Kaffee\\
bei dem Frühstück bis halb 8.}

\strophe{Geht es doch dann an die Arbeit\\
nehmen sie Stift und Papier\\
verfassen Resolutionen\\
dafür sind wir nunmal hier.\\
}

\refrain{(Pfeifen)}

\strophe{
Ja das Plenum ist gelaufen\\
alle nähen schon nicht mehr\\
denn die Katzen die sie machten\\
liegen hier schon kreuz und quer.}

\strophe{Ja die KoMa ist vorüber\\
nur noch eine Frage plagt\\
doch wir können euch beruhigen\\
AK Pella hat getagt.\\
}

\refrain{(Pfeifen)}



\lied{Ich will zurück zur ZKK}{Westerland}{2015}

\strophe{
Jeden Tag sitz ich im Hörsaal\\
und ich hör den Profen zu.\\
Sitz auf unbequemen Stühlen\\
und die Augen fall'n mir zu!}

\strophe{
Diese eine Tagung sollt nie zuende geh'n!\\
Warum muss ich schlafen geh'n?}

\strophe{
Und die vielen lieben Engel\\
rennen hektisch hin und her,\\
kümmern sich um unser Essen\\
und dann noch um vieles mehr!}

\strophe{
Diese eine Tagung sollt nie zuende geh'n!\\
Warum muss ich schlafen geh'n?}

\refrain{
Oh ich habe solche Sehnsucht,\\
ich wünscht ich wär' wieder da.\\
Ich will zurück ins schöne Aachen,\\
ich will zurück zur ZKK!}

\strophe{
Wie oft saß ich in nem AK,\\
wie oft stand ich am Buffet,\\
wie oft musst ich Mate kippen,\\
damit ich noch aufrecht steh'!}

\strophe{
Diese eine Tagung sollt nie zuende geh'n!\\
Wann darf ich nochmal hingeh'n?}

\strophe{\textsl{Refrain}}

\strophe{
Es ist zwar etwas selten,\\
dafür sind wir vollzählig.\\
Und ich weiß jeder einzelne\\
ist genauso schräg wie ich.}

\strophe{\textsl{Refrain (\(4\times\) "`ich will zurück"' vor "`zur ZKK"')}}

\lied{Überraschungsreso}{Happy Birthday to you}{1924}

\refrain{
Überraschungsreso.\\
Überraschungsreso.\\
Hier im Plenum gibt's 'ne kleine\\
Überraschungsreso.}



\cleardoublepage
\phantomsection
\addcontentsline{toc}{chapter}{Weihnachten}
\lied{O du Integral}{O du Fröhliche}{2012}

\strophe{
O du Integral, O du Rie--e--mann,\\
Der du uns gabst das Integral.\\
Geht Obersumme\\
Gen Untersumme\\
Nach dx, nach dx, integrieren.}

\strophe{
O du Integral, o du Le--e--besgue,\\
Der du uns gabst ein Maß dazu\\
Miss die Funktionen\\
Es wird sich lohnen\\
Nach d\(\mu\), nach d\(\mu\), integrieren.}

\strophe{
O du Integral, o du Ableitung\\
Denn für stet’ge f von x gilt\\
Hast du integriert\\
Und dann diff’renziert\\
Bekommst du, bekommst du doch f zurück.}




\lied{Oh Mein Raum}{Oh Tannenbaum}{2012}

\strophe{
Oh Vektorraum, oh Vektorraum,\\
du lebst über 'nem Körper.\\
Jede der Basen spannt dich auf,\\
nicht immer findet man sie auch.\\
Oh Vektorraum, oh Vektorraum,\\
du lebst über 'nem Körper.}

\strophe{
Oh Banachraum, oh Banachraum,\\
wie schön sind deine Normen!\\
Sie führ'n nicht nur zur Metrik hin,\\
auch Cauchy-Folgen bleiben drin.\\
Oh Banachraum, oh Banachraum,\\
wie schön sind deine Normen!}

\strophe{
Oh Hilbertraum, oh Hilbertraum,\\
du hast ein inn'res Produkt.\\
Die Physiker, sie lieben dich.\\
Für sie bist du einzigartig.\\
Oh Hilbertraum, oh Hilbertraum,\\
du hast ein inn'res Produkt.}

\strophe{
Oh Hausdorffraum, oh Hausdorffraum,\\
man kann hier Punkte trennen.\\
Ihre Umgebung' sind disjunkt,\\
in jeder liegt ein and'rer Punkt.\\
Oh Hausdorffraum, oh Hausdorffraum,\\
man kann hier Punkte trennen.}




\lied{Morgen Ersties wird's was geben}{Morgen Kinder wird's was geben}{2012}

\strophe{
Morgen Ersties wird's was geben,\\
morgen werden wir uns freu'n!\\
Welch ein Schnauben, welch ein Stöhnen\\
wird in uns'rem Hörsaal sein.\\
Einmal werdet ihr noch wach\\
heissa dann ist euer Tag.}

\strophe{
Wie wird dann der Kopf euch rauchen,\\
wie wird dann die Birne glüh'n!\\
Formeln werdet ihr dann brauchen,\\
Sätze müsst ihr dann bemüh'n!\\
Wisst ihr noch wie letztes Jahr\\
es bei der Matura war.}

\strophe{
"`Welch ein schlimmer Tag ist morgen?"'\\
Das fragt ihr euch nun gewiss!\\
Macht euch schon sehr viele Sorgen,\\
habt schon alle ziemlich Schiss.\\
Einmal werdet ihr noch wach\\
heissa dann ist Prüfungstag!}

\strophe{
\textit{Ich} werd' alles korrigieren\\
werde jeden Fehler seh'n!\\
Werde sie schön rot markieren,\\
Keiner bleibt hier ungeseh’n.\\
Sind davon zu viele da,\\
seh’n wir uns dann nächstes Jahr\ldots}




\lied{Last Lecture}{Last Christmas}{2012}

\refrain{
Last lecture I gave you a proof\\
but the very next day you showed me my fail\\
This time to save me from tears,\\
I’ll present you something special.}

\strophe{(repeat)}

\strophe{
My sweet proof is still wrong.\\
The next approach will surely come along\\
Tell me reader, do you understand it?\\
Well, neither do I, I don't comprehend it.}

\strophe{
The paper I wrapped it up and sent it\\
With a note saying ``q.e.d.'', I meant it!\\
Now I know what a fool I've been\\
But reading it now, I know I will fail again.}

\refrain{
Last lecture I gave you a proof\\
but the very next day you showed me my fail\\
This time to save me from tears,\\
I’ll present you something special.}

\strophe{(repeat)}

\strophe{
Oooohhhh\\
Oh oh, reader}

\strophe{
The idea was great, and the error so small\\
And maybe the reader will not find it at all.\\
Let's see if it will stay unrecognized\\
And I will pass the course with it.}


\newpage
~
\cleardoublepage
\pagestyle{empty}
\thispagestyle{plain}
~
\cleardoublepage
\end{document}
